\documentclass[12pt]{article}
\usepackage{amsmath}
\usepackage{graphicx}
\usepackage{caption}
%\usepackage{refstyle}
\usepackage{csvsimple}
\author{\large Shrikant Hamand \\[12pt] Mis No. : 112103048}
\date{\today}
\begin{document}
\raggedright
\title{\huge \bf DTL Assignment 4}
\maketitle
\pagenumbering{gobble}
\newpage
\tableofcontents
\newpage
\pagenumbering{roman}
\section{ Using Section and Subsection}
\vspace{5mm}
\subsection{Chapter 1 : Abstract Machines}
One of the most general concepts employing abstraction is the \textit{\textbf{abstract machine}} .
\paragraph{In this chapter, we will see how this concept is closely related to the programming languages.} We will also see how, without requiring us to go into the specific details of any particular implementation, it allows us to describe what an implementation
of a programming language is. To do this, we will describe in general terms what
is meant by the interpreter and the compiler for a language. Finally, will see how
abstract machines can be structured in hierarchies that describe and implement complex software systems.
\subsubsection{ Concepts of Abstract Machine and of Interpreter}
\subparagraph{ the context of this book, the term \textit{“machine”} refers clearly to a computing machine. As we know, an electronic, digital computer is a physical machine that executes algorithms which are suitably formalised so that the machine can \textit{“understand”} them. Intuitively, an abstract machine is nothing more than an abstraction of the concept of a physical computer.}
\subsubsection{The Interpreter}
Clearly the interpreter must perform the operations that are specific to the language it is interpreting, L . However, even given the diversity of languages, it is possible to discern types of operation and an \textit{“execution method”} common to all interpreters.
\subsubsection{An Example of an Abstract Machine: The Hardware Machine}
From what has been said so far, it should be clear that the concept of abstract machine can be used to describe a variety of different systems, ranging from physical machines right up to the World Wide Web.\\
As a first example of an abstract machine, let us consider the concrete case of a
conventional physical machine such as that in Fig. 1.3. It is physically implemented using logic circuits and electronic components. Let us call such a machine MHL H and let L H be its machine language.
\subsection{Chapter 2 : How to Describe a Programming Language}
A programming language is an artificial formalism in which algorithms can be expressed. For all its artificiality, though, this formalism remains a language. Its study can make good use of the many concepts and tools developed in the last century in linguistics (which studies both natural and artificial languages). Without going into great detail, this chapter poses the problem of what it means to “give” (define or understand) a programming language and which tools can be used in this undertaking
\subsubsection{Levels of Description}
In a study which has now become a classic in linguistics, Morris [6] studied the
various levels at which a description of a language can occur. He identified three
major areas: grammar, semantics and pragmatics.
Grammar is that part of the description of the language which answers the question: which phrases are correct? Once the alphabet of a language has been defined as a first step (in the case of natural language, for example, the Latin alphabet of 22 or 26 letters, the Cyrillic alphabet, etc.), the lexical component, which uses this alphabet, identifies the sequence of symbols constituting the words (or tokens) of the language defined. 
\subsubsection{Grammar and Syntax}
We have already said that the grammar of a language first establishes the alphabet
and lexicon. Then by means of a syntax, it defines those sequences of symbols corresponding to well-formed phrases and sentences (or to “sentences” in short). Clearly, at least from the viewpoint of natural language, the definition of the (finite) alphabet is immediate.
\newpage
\pagenumbering{arabic}
\section{ Maths Paper}
\vspace{10mm}
\begin{center}
\huge \bf {College Of Engineering Pune} 
\linebreak \small{(An Autonoumous Institute Of Government Of Maharashtra)}\\
\vspace{10mm}
\large \textbf{End Semester Examination} \\  Ordinary Differential Equation And Multivariate Calculus
\\ (MA-16001)
\end{center}
\par\noindent\rule{\textwidth}{1pt}
\\
\vspace{5mm}
\noindent \textbf{Date :} \today \hfill \textbf {Duration :} 1 hour
\\ \textbf {Branches :} All \hfill  \textbf {Max marks} : 60
\\ \textbf {Programme :} S.Y B.Tech \hfill \textbf {Semester :} I
\linebreak
\linebreak
\\ \textbf {Name:} Shrikant Hamand \hfill \textbf{MIS.NO : }112103048
\linebreak
\linebreak
\textbf{Division :} 1 \hfill \textbf{Batch :}  S4
\linebreak
\\ \textbf{Instructions:}
\linebreak
\\ \textbf{1.} All questions are compulsory.
\\ \textbf{2.} All symbols have their usual meanings.
\\ \textbf{3.} Figures to right indicate course outcomes and full marks.
\\ \textbf{4.} Mobile phones and programmable calculators are not allowed.
\\ \textbf{5.} Writing anything on question paper , exchange of stationary, calculator is strictly not allowed.
\\ \textbf{6.} Write all subparts of question of question together.
\newpage
\newpage
\subsection*{Q1. Solve any three questions }
\textbf{a)}Write  $y_p$ (particular solution) for the differential equation 
\begin{align*}
y''+7y'+12y=cos(6x).
\end{align*}
\\ \textbf{b)}A Capacitor = 0.2 farad in series with a resistor R= 20 Ohs is charged from a source $E_o$ = 24 V. Find the voltages v(t) on thee capacitor, assuming that at t=0 the capacitor is completely uncharged.
\linebreak
\\ \textbf{c)}Solve the integration 
\[ \int_{a}^{b} x^2 \,dx \]
\linebreak
\\ \textbf{d)} Solve the differential equation y''+y'=$2x^2+e^2x$, using the method of Undetermined Coefficients.\\
\subsection*{Q2. Solve the following equations}
\vspace{5mm}
\textbf{(1)} Which of the following sequences converge and which diverge? Find the limit of each convergent sequence and justify your answers.\linebreak

(i) $a_n = (-1)^n(1 - \dfrac{1}{n})$ \hspace{15mm} (ii) $a_n = \dfrac{\ln n}{n^{\frac{1}{n}}}$\\
\vspace{5mm}
\textbf{(2)} For any $\triangle ABC$, prove that -\\
$$a \sin (B-C) + b \sin (C - A) + c \sin (A-B) = 0$$
\linebreak

\textbf{(3)}Find the value of $$\sqrt{3} \csc (20^{\circ}) - \sec (20^{\circ})$$.
\linebreak

\textbf{(4)}Find the value of $f(x) = \dfrac{tan(3x) \ln (x+1)}{\sqrt[3]{\sin^6 x}}$ when x tends to 0.
\vspace{5mm}
\linebreak
\textbf{(5)} Find $\dfrac{dy}{dx}$ for $ y = e^{6 \log_e (x-1)}$, $ x > 1$.\\

\subsection*{Q3. Solve the following equations}
\textbf{(a)} Solve the following differential equation.
\begin{align}
 y'' + by' + cy = 0  
\end{align}
\textbf{(b)} Solve the following Homogeneous differential equation.
\begin{equation*}
 y''+y'=2x^2+e^2x
\end{equation*}
\textbf{c)}
\begin{align*}
x&=y           &  w &=z              &  a&=b+c\\
2x&=-y         &  3w&=\frac{1}{2}z   &  a&=b\\
-4 + 5x&=2+y   &  w+2&=-1+w          &  ab&=cb
\end{align*}
\textbf{(d)}
Solve the summation:\\
\[ \sum_{n=1}^{\infty} 2^{-n} = 1 \]


\textbf{(e)} Obtain the general solution (or particular solution) of the following differential equations. 
\begin{equation*}
y_0 = 2 sec 2y
\end{equation*}


\subsection*{Q4. Solve the following questions}
\textbf{a)}Solve the following determinants
\subsubsection*{(i)}
$\begin{vmatrix}
1 & 2 & 3\\
4 & 5 & 6\\
7 & 8 & 9
\end{vmatrix}$
+
$\begin{vmatrix}
4 & 5 & 9\\
6 & 2 & 6\\
3 & 8 & 7
\end{vmatrix}$

\subsubsection*{(ii)}
$\begin{vmatrix}
1 & 2 & 3\\
4 & 5 & 6\\
7 & 8 & 9
\end{vmatrix}$
*
$\begin{vmatrix}
4 & 5 & 9\\
6 & 2 & 6\\
3 & 8 & 7
\end{vmatrix}$
\vspace{5mm}

\textbf{b)}Find $\det (AB)$ if  $A = \begin{bmatrix}
\sin x & \cos x \\ \cos x & -\sin x
\end{bmatrix}$  and  $B = \begin{bmatrix}
2 & \tan x \\ \cos x & 0
\end{bmatrix}$.

\vspace{5mm}
\textbf{c)}Let k be a number. Then the matrix $A = \begin{bmatrix}
k & 0 & \cdots & 0\\
0 & k & \cdots & 0\\
\vdots & \vdots & \ddots & \vdots\\
0 & 0 & 0 & k
\end{bmatrix}$ is called as ?


\newpage
\section{ Inserting Images }
\begin{figure}[h]
\includegraphics[width=1.0\textwidth]{landscape}
\caption{Kakashi Hatake}
\label{fig: Landscape}
\end{figure}

%Image in \figref{Wheat Field} is a Wheat Field at the time of sunset under black clouds sky.
\newpage
\section{ Inserting Table }
\subsection{Adding Table Manually}
\begin{table}[h]
\begin{center}
\caption{Marks of students in T1}
\label{tab:Table 1} 
\vspace{5mm}
\begin{tabular}{|p{2cm}|c|c|c|c|}
\hline
\textbf{Name} & \textbf{FCS} & \textbf{ODEMC} & \textbf{PPL} & 				\textbf{DSA}\\
\hline
Himanshu  & 18 & 13 & 14 & 19\\
Dhananjay & 15 & 16 & 12 & 17\\
Harshal & 14 & 17 & 15 & 14\\
Hariom & 17 & 14 & 13 & 15\\
Aditya & 11 & 19 & 16 & 16\\
\hline
\end{tabular}
\end{center}	
\end{table}


\subsection{Adding Table from .csv file}
\begin{table}[h]
\centering
\csvautotabular[]{data.csv}
\end{table}

\newpage
\section{ Making Bibliography}
\subsection{Simple Method}
Moving Average Crossover: After graphing, two 
moving averages based on separate time periods tend to cross, 
which is known as a moving-average crossover ~\cite{Paper}. A quicker 
moving average and a slower moving average are used in this 
indication (or more). The shorter moving average (short-term) ~\cite{Latex}
can be 5, 10, or 15 days, while the longer-term moving ~\cite{aa,agarwal}
average might be 100, 200, or 250 days. Since it only 
evaluates prices over a short period of time, a short-term 
moving average is speedier and more responsive to daily 
price changes ~\cite{agarwal,Kshir,rucha}


\begin{thebibliography} {}

\bibitem {aa}agarwal Ugale.,N. Bhatt.,Enginnering Physics SMA,978-1-6654-1703-7/21.

\bibitem{agarwal} S. Agarwal.,N. Bhatt.,Enginnering Physics SMA,978-1-6654-1703-7/21.

\bibitem{Latex} Latex,Latex ,IEEE

\bibitem{Kshir} S. Kshirsagar.,N. Bhatt.,Enginnering Physics SMA,978-1-6654-1703-7/21.

\bibitem{rucha} S. Agarwal.,S. Kshirsagar.,Enginnering Physics SMA,978-1-6654-1703-7/21.

\bibitem{Paper} ,Research Paper,2022,IEEE.
\end{thebibliography}
\newpage
\subsection{Bibtex Method}

The intensification of senior population has been a concern worldwide [1].
Compared to the younger generations, the elderly always experience critical
challenges in their activity of daily living (ADL) [2-4]. Rather disconcertingly,
most of them are suffering degeneration and geriatric issues, which lead them to be dependent on a long-term medication intake. Consequently, this is causing them to be more vulnerable against potential harms in their living environment [2].
Therefore, many elders prefer to reside in healthcare institutions rather than taking the risk of being less attended to and supported while being at home.\\ However, the absence of standardization management for elderly care institutions in the industry requires some working up. Currently, elderly centres have their particular regulations on administration which not only generates management gaps in operation but also slows down the development pace of the entire industry.\\
Due to the many issues in geriatric care, routine medication plays an essential
role in the ADLs of many elderly. In this case, medicine mismanagement becomes
a lethal factor for the elderly in care centres. However, many centres are still
equipped with manually performed medicine closets Fig. 1, which post hidden
dangers of medicine retardation, mismatch or over dosage perhaps inadvertently or
by careless caregivers. s


\begin{thebibliography} {}

\bibitem{Boney96} Boney, L., Tewfik, A.H., and Hamdy, K.N., ``Digital
Watermarks for Audio Signals," \emph{Proceedings of the Third IEEE
International Conference on Multimedia}, pp. 473-480, June 1996.
\bibitem{MG} Goossens, M., Mittelbach, F., Samarin, \emph{A LaTeX
Companion}, Addison-Wesley, Reading, MA, 1994.
\bibitem{HK} Kopka, H., Daly P.W., \emph{A Guide to LaTeX},
Addison-Wesley, Reading, MA, 1999.
\bibitem{Pan} Pan, D., ``A Tutorial on MPEG/Audio Compression," \emph{IEEE
Multimedia}, Vol.2, pp.60-74, Summer 1998.
\end{thebibliography}
\end{document}