\documentclass[12pt]{article}
\usepackage{graphicx}


\author{\bf Shrikant V. Hamand \\[10pt] Mis No. : 112103048}
\date{\today}

\begin{document}

\begin{figure}
\centering
\includegraphics[scale=1]{coep_new_logo}
\end{figure}

\title{\huge \bf DTL Assignment 1}
\maketitle
\newpage
\tableofcontents
\newpage
\pagenumbering{gobble}
\newpage
\pagenumbering{arabic}
\section{Courses Syllabus}

\vspace{1cm}
\subsection{Ordinary Differential Equations and Multivariate Calculus}

\vspace{5pt}
\paragraph{Unit I :} Review of first order differential equations, Reduction of order, linear differential 
equations, homogeneous higher order linear differential equations, non-homogeneous higher 
order linear differential equations with constant coefficients and reducible to differential 
equations with constant coefficients (method of undetermined coefficients and method of 
variation of parameters), systems of differential equations, applications to orthogonal 
trajectories, mass spring systems and electrical circuits.\\[5pt]

\paragraph{Unit II :} Laplace Transforms, its properties, Unit step function, Dirac delta functions, 
Convolution Theorem, periodic functions, solving differential equations using Laplace 
transform.\\[5pt]
\paragraph{Unit III :} Functions of several variables, level curves and level surfaces, partial and directional 
derivatives, differentiability, chain rule, local extreme values and saddle points, constrained 
optimization. 
\vspace{1cm}
\subsection{Innovation and Creativity}
\vspace{5pt}
\paragraph{Contents}
\subparagraph{Introduction to concepts of creativity / invention / innovation and their importance in present 
knowledge world. Components of the creative process, Analogy/model to represent the creative 
process.}
\subparagraph{Understanding persons Creative potential. Blockages in practicing creative process – Mindset 
and belief systems. Myths and misconceptions about creativity.}
\subparagraph{Practical Tips to discover and apply one's creative potential, remove blockages, deal with 
external factors. Importance of synergistically working in a team. Harnessing creativity from 
nature.}
\subparagraph{\textit{Idea conception,Idea Brainstorming sessions, Idea Evaluation, Protection/Patent 
review,Principles of innovation, Review of systematic strategies and methods for 
innovation,Innovation case study,Review of Idea/Prototype /Product and Market Plan.}}
\subparagraph{Applications Exercise / Assignment: at the end of the course, the student will create teams, 
presents their innovative ideas, and applies their learning in practice.}

\subsection{Development Tools Laboratory}
\vspace{5pt}
\textbf{Course Contents} \\[5pt] 
\textbf{LaTEX:} Basic syntax, compiling and creating documents; Document structure, sections, 
paragraphs; packages, Math, Adding Images, Drawing images (using tools like Inkscape) Table 
of contents; Source code, graphs (using tools like Graphviz), Adding references, different 
templates, IEEE format, Bibliography
\\[5pt]
\textbf{Shell Programming:} Introduction to Linux commands, concept of shell, shell variables, 
getcwd() and pwd; Introduction to shell programming features: Variables declaration \& scope, 
test, return value of a program, if-else and useful examples, for and while loop, switch case; 
Shell functions, pipe and redirection, wildcards, escape characters; Awk script: Environment and 
workflow, syntax, variables, operators, regular expressions, arrays, control flows, loops, 
functions, output redirections
\\[5pt]
\textbf{GIT:} Creating a project using git locally, add, commit, status, diff; branch and merge, GIT: 
cloning a remote repo, working with a remote repo – git push, pull, fetch; creating issues and pull 
requests; working on a project in a distributed fashion
\vspace{1cm}

\end{document}